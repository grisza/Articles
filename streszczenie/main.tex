\documentclass[10pt,a4paper]{article}
\usepackage[utf8]{inputenc}
\usepackage[OT4]{fontenc}
\usepackage{enumerate}
\usepackage{polski}
\usepackage[polish]{babel}
\usepackage{microtype}
\usepackage[short]{datetime}
\usepackage{xparse}
\usepackage{calc}
\usepackage{fancyhdr}
\usepackage{graphicx}
\usepackage[unicode]{hyperref}% unicode is required for unicode pdf metadata
\usepackage{etoolbox}


\usepackage{ragged2e}
\author{Grzegorz Glonek}

\begin{document}	

	\begin{FlushLeft}
mgr inż. Grzegorz Glonek \hfill 	Łódź, \today \\					
	Promotor: dr hab. inż. Adam Wojciechowski
\end{FlushLeft}

\vspace*{2cm}
\Centering
STRESZCZENIE ROZPRAWY DOKTORSKIEJ\\
\vspace*{0.5cm}
\textbf{Hybrydowa metoda sledzenia ruchu człowieka w czasie rzeczywistym}\\

\vspace*{1.5cm}
\justify
\textbf{\underline{Teza:}} \textbf{Zastosowanie autorskiej, hybrydowej metody śledzenia ruchu kończyn człowieka, łączącej dane pochodzące z~sensora głębi i~sensorów inercyjnych, uwzględniającej kontekstowe charakterystyki pracy urządzeń, pozwala na bardziej precyzyjne śledzenie ruchu.}

	\begin{FlushLeft}
	W rozprawie wyróżniono następujące cele szczegółowe:
	\begin{enumerate}[1. ]
		\item zdefiniowanie charakterystyk działania i ograniczeń zastosowanych urządzeń pomiarowych;
	
		\item opracowanie i zastosowanie metod kompensacji ujawnionych i mierzalnych ograniczeń zastosowanych urządzeń pomiarowcyh;
			\item opracowanie autorskiej metody łączenia sygnałów pochodzących z urządzeń pomiarowych: kontrolera Microsoft Kinect i czujników inercyjnych działającej z częstotliwością zbliżoną do częstotliwości pracy kontrolera Kinect tj. $30Hz$ i uwzględniającej kompensację ujawnionych ograniczeń zastosowanych urządzeń pomiarowych;		
		\item zastosowanie opracowanej metody łączenia sygnałów do śledzenia kończyn wykonujących ruch w tempie charakterystycznym dla ćwiczeń rehabilitacyjnych.
	\end{enumerate}

\textbf{Ad. 2}: W celu opracowania metod kompensacji ograniczeń działania wykorzystywanych urządzeń pomiarowych, został wykonany szereg eksperymentów mających na celu ich ujawnienie i opisanie. Wykazane ograniczenia były zazwyczaj pominięte w oficjalnej dokumentacji wykorzystywanych urządzeń. Temat związany ze zbadaniem istotnych ograniczeń i niedokładności w działaniu wspomnianych urządzeń nie był także przedmiotem kompleksowych badań w świetle znanej mi literatury przedmiotu.

\textbf{Ad. 3}: Dotychczas opisywane w literaturze metody łączenia danych z kontrolera Microsoft Kinect oraz z urządzeń inercyjnych opierały się na informacji o położeniu wybranych stawów oraz na przetwarzaniu tych informacji za pomocą filtru Kalmana. Niewiele znanych mi publikacji wykorzystuje inny mechanizm łączenia danych niż wspomniane filtry Kalmana. W żadnym z artykulów nie znalazłem opisu wykorzystania innych danych, do łączenia pomiarów z obu źródeł, niż położenia przestrzennego wybranych stawów.

\textbf{Ad. 4}: W celu weryfikacji działania opracowanej metody, została ona przetestowana na zestawie ruchów ręki wykonywanych w kierunkach mogących zostać określone jako wymagające dla tych urządzeń pomiarowych. Oznacza to, że w trakcie ich wykonywania zostały uwypuklone ograniczenia kontrolera Microsoft Kinect lub czujników inercyjnych. Ruchy były wykonywane w tempi charakterystycznym dla ćwiczeń rehabilitacyjnych, tak aby nie przekroczyć zdolności rejestracyjnych wspomnianych urządzeń pomiarowych. Gdyby wykorzystać ruchy wykonywane z dużą prędkością, które są charakterystyczne dla treningu sportowego, wówczas należałoby zastosować inne urządzenia pomiarowe charakteryzujące się wyższą częstotliwością pracy. 

Aby zrealizować cele szczegółowe 1-4, wymienione powyżej, koniecznym okazało się m.in.:
\begin{enumerate}[1. ]
	\item kompleksowe zbadanie charakterystyk działania kontrolera Kinect oraz czujników inercyjnych;
    \item opracowanie metod kompensacji ujawnionych ograniczeń w działaniu wykorzystywanych urządzeń pomiarowych;
	\item opracowanie autorskiej metody łączenia danych z kontrolera Microsoft Kinect i czujników inercyjnych w oparciu o informacje o obrotach wybranych kości;
\end{enumerate}

Rozprawa została podzielona na pięć rozdziałów. W rozdziale 1 (str. 11) zostały przedstawione i usystematyzowane podstawowe pojęcia wykorzystywane w dalszej części pracy, sformułowano problem badawczy, cele szczegółowe rozprawy oraz jej tezę, a także uzasadniono motywację podjęcia badań. Rozdział 2 (str. 17) przedstawia aktualny stan wiedzy dotyczący obszaru prowadzonych badań, a także wyniki badań własnych związanych z określeniem charakterystyk działania wykorzystywanych urządzeń pomiarowych. Zagadnieniami poruszonymi w tym rozdziale są systemy śledzenia ruchu, ich taksonomia oraz charakterystyka działania (roz. 2.1), komputerowe modele postaci ludzkiej (roz. 2.2), charakterystyki urządzeń pomiarowych: kontrolera Microsoft Kinect oraz czujników inercyjnych (roz. 2.3). Ostatni z podrozdziałów (roz. 2.4) przedstawia analizę literatury przedmiotu dotyczącą metod łączenia sygnałów pochodzących z optycznych, bezmarkerowych, systemów śledzenia ruchu z sygnałami urządzeń inercyjnych. Podrozdział 2.3 jest realizacją pierwszego celu szczegółowego rozprawy doktorskiej. W rozdziale 3 (str. 79) opisana została autorska, hybrydowa, metoda śledzenia ruchu człowieka łącząca dane pochodzące z kontrolera Microsoft Kinect oraz czujników inercyjnych. Kolejne podrozdziały poświęcone są poszczególnym etapom przetwarzania danych w prezentowanej metodzie. W pierwszym podrozdziale (3.1) opisany jest format danych dostarczanych przez urządzenia pomiarowe, a także opis autorskiego urządzenia pomiarowego wykorzystującego czujniki inercyjne oraz platformę Arduino. Podrozdział 3.2 opisuje autorski algorytm kalibracji czujników inercyjnych oraz procedurę inicjalizacji filtru Madgwicka wykorzystywanego do łączenia sygnałów pochodzących z żyroskopu oraz z akcelerometru. Podrozdział 3.3, będący realizacją drugiego celu szczegółowego, przedstawia metody kompensacji niedoskonałości urządzeń inercyjnych opisanych w rozdziale 2.3. Kolejny podrozdział (3.4) opisuje wykorzystaną metodę na określenie przesunięcia czasowego pomiędzy sygnałami pochodzącymi z czujników inercyjnych oraz z kontrolera Kinect i ich synchronizacji. Ostani podrozdział (3.5) zawiera opis klasyfikacji odbieranych danych ze względu na ich wiarygodność, a tym samym na wybór metody łączenia otrzymanych sygnałóW które to zostały opisane w drugiej części tego podrozdziału. Informacje zawarte w rozdziale 3.5 są realizacją trzeciego z założonych celi szczegółowych. Realizację czwartego z przyjętych celi szczegółowych opisuje rozdział 4 (str. 103) w którym zawarte są opisy przeprowadzonych eksperymentów oraz dyskusja dotycząca otrzymanych wyników. Rozdział 5 (str. 123), będący ostatnim rozdziałem zasadniczym, stanowi podsumowanie rozprawy, zreasumowano w nim wyniki przeprowadzonych badań, sformułowano wnioski oraz wskazano możliwe kierunki rozwoju badań.

Ponadto w pracy znalazły sie dwa rozdziały dodatkowe  majace na celu przedstawienie informacji pomocnych w analizie przedstawionego tematu. Pierwszym z nich (dodatek A, str. 131) jest opis podstawowych filtrów stosowanych do łaczenia sygnałów, a wykorzystywanych w przytoczonych w niniejszej pracy artykułach. Filtrami opisanymi w tym dodatku, w kolejności ich opisu, są filtry Kalmana: liniowy, rozszerzony i bezśladowy, filtr komplementarny oraz filtry Mahoney’a i Madgwicka. W drugim dodatku (dodatek B) znajduje się ogólny opis wyznaczania i zastosowania wariancji Allana, będacej narzędziem pozwalającym na analizę charakterystyki szumów występujących w elektronicznych urządzeniach pomiarowych, m.in. w czujnikach inercyjnych.

Dodatkowo rozprawę rozpoczyna wykaz użytych skrótów i symboli, a kończą bibliografia oraz spisy tabel i rysunków

W części związanej z przedstawieniem aktualnego stanu, zaprezentowano przegląd istniejących systemów śledzenia ruchu. Szczególnie dużo uwagi poświęcono dwóm rodzajom takich systemów: optycznemu bezmarkerowemu oraz inercyjnemu. Rozszerzony opis tych właśnie systemów spowodowany jest tym, że autorska metoda śledzenia ruchu wykorzystuje urządzenia charakterystyczne własnie dla tych dwóch rodzajów systemów. Z tego też powodu jest szczególnie istotne aby przedstawić szerzej czytelnikowi, najważniejsze cechy tych systemów. Istotnym punktem rozprawy jest analiza budowy i działania kontrolera Kinect oraz czujników inercyjnych z uwzględnieniem ich ograniczeń. W toku badań własnych autora zidentyfikowane i opisane zostały następujące charakterystyki:

\begin{enumerate}
	\item wrażliwość kontrolera Kinect na występowanie dodatkowych źródeł światła podzczerwonego w tym światła słonecznego lub drugiego kontrolera (str. 55);
	\item brak pełnej informacji o orientacji przestrzennej kości w danych otrzymywanych z kontrolera Kinect (str. 56) objawiający sie brakiem informacji o rotacji w sensie kątów Eulera (\emph{ang. roll}, str. 47);
	\item występowanie okluzji stawów w trakcie prowadzenia pomiarów przez Kinecta w tym m.in. okluzji spowodowanej zbyt dużym obrotem sylwetki do płaszczyzny obserwacji Kinecta (str. 57);
	\item zmiana dokładności szacowania odległości pomiedzy kontrolerem Kinect a użytkownikiem w zależności gdzie użytkownik się< znajduje (str. 60);
	\item wpływ temperatury urządzenia inercyjnego na jego pomiary (str. 68) wynikający z budowy tych czujników w architekturze MEMS (str. 64);
	\item brak pełnej informacji o orientacji przestrzennej modułu inercyjnego (str. 69) objawiający sie brakiem informacji o skręcie w sensie kątów Eulera (\emph{ang. yaw}, str. 47);
\end{enumerate}

W części związanej z aktualnym stanem wiedzy, przedstawiony został także przegląd metod związancyh z łączeniem optycznych bezmarkerowych systemów śledzenia ruchu z systemami inercyjnymi. Analiza wyróżnionych w tej części pozycji bibliograficzncyh pozwala na wysunięcie następujących spostrzeżeń:

\begin{itemize}
\item najczęstszą metodą łączenia ze sobą sygnałów pochodzących z bezmarkerowych optycznych systemów śledzenia ruchu z sygnałami pochodzącymi z systemów inercyjnych jest filtr Kalmana,
\item sygnały łączone są se zosbą na podstawie informacji o położeniu wybranych stawów w przestrzeni,
\item autorzy nie uwzględniają w swoich metodach korekty błędów w danych źródłowych
\item deklarowany błąd szacowania pozycji najdokładniejszej z opisanych metod wynosi około $2.2cm$ (metoda wykorzystana jako odniesienie do metody autorskiej. W prowadzonych badaniech własnych nie udało się osiągnąć deklarowanej dokładności).
\end{itemize}

Na podstawie powyższych zostały przygotowane metody kompensujące błędy związane z niewłaściwym określaniem odległości przez kontroler Kinect (str. 92) oraz z wpływem temperatury na pomiary modułów inercyjnych (str. 90), na podstawie formuł matematyczncyh charakteryzujących te błędy. Z uwagi na to, że brakujące informacje o orientacji przestrzennej dotyczą różnych osi, informacje te są uzupełniane w trakcie łączenia danych. Występowanie okluzji danych zostało wykorzystane jako jeden ze wskaźników wiarygodności pomiarów konrolera Kinect. Badając wartość kąta obrotu użytkownika względem kontrolera kinect wraz ze stabilnością tego pomiaru oraz status śledzenia wybranych stawów zdefiniowano kryteria czy można wykorzystać pomiary uzyskane z Kinecta w procesie łączenia danych czy nie.

Zaprezentowana w pracy metoda zawiera nowe podejscie do łaczenia ze soba danych z czujników inercyjnych i opisu modelu szkieletowego udostepnianego przez kontroler Kinect. Metoda ta, w odróżnieniu od już istniejących, łączy sygnały źródłowe w oparciu o informację o orientacji przestrzennej wybranych kości przyjętego modelu szkieletowego człowieka. Wykorzystanie charakterystyk obu urządzeń pomiarowych pozwala w istotny sposób naprawić dane źródłowe, które będa podlegały złączeniu, a także określeniu wiarygodności a co za tym idzie istotności poszczególnych danych. Dzięki temu samo złączenie ze sobą danych odbywa się w mniej złożony sposób niż ma to miejsce w metodach opisanych w dostępnej literaturze przedstawionej w rozdziale 2. 

Założeniem rozprawy było opracowanie i przedstawienie autorskiej, hybrydowej, metody śledzenia ruchu człowieka łączącej ze sobą sygnały kontrolera Kinect oraz czujników inercyjnych, uwzględniając i kompensując ograniczenia tychże urządzeń pomiarowych. Sformułowane w rozdziale 1 (str. 11), cele szczegółowe zostały osiągnięte, a postawiona teza potwierdzona. W pracy wykazano, że nowatorskie podejście łączenia danych ze wspomnianych urządzeń pomiarowych w oparciu o informację o orientacji przestrzennej z jednoczesną kompensacją znanych, mierzalnych błędów zawartych w danych źródłowych, zmniejsza błąd szacowania pozycji wybranych stawów w stosunku do tradycyjnych metod znanych z literatury.

W rozprawie zwrócono uwagę na niedoskonałości wykorzystywanych urządzeń pomiarowych, mających widoczny wpływ na dokładność i wiarygodność otrzymywanych danych. Dzięki temu udało się określić konteksty wykonywanego ruchu, jak na przykład obrót sylwetki śledzonej postaci względem kontrolera Kinect (str. 57) czy odległość od urządzenia pomiarowego w jakiej znajdują się wybrane stawy (str. 60), które mają wpływ na uzyskiwane pomiary. Jednocześnie zostały zaproponowane metody kompensujące błedy występujące w uzyskanych pomiarach. Wskazano również jak błędy wynikające z kontekstu wykonywanego ruchu wpływają na poszczególne pomiary, na przykład w zależności od odległości między wybranym stawem a kontrolerem Kinect, błąd oszacowania tej odległości przez urządzenie pomiarowe waha się od $10cm$ niedoszacowania do około $15cm$ przeszacowania (str. 62). Błędy te zostały opisane za pomocą formuł matematycznych lub zestawu cech charakteryzujących je co pozwala na ich łatwą detekcję i kompensację.

Zaproponowaną metodę przetestowano na zestawie ćwiczeń, których ruch odbywał się zarówno w płaszczyźnie nie sprawiającej trudności w śledzeniu ruchu dla żadnego z urządzeń, jak i w takich płaszczyznach, w których ujawnione wcześniej niedoskonałości, zostały szczególnie uwypuklone. Pozwoliło to na sprawdzenie proponowanej metody w sytuacji kiedy dane z obu urządzęń pomiarowych są wystarczająco wiarygodne aby można je było ze sobą połączyć, jak i w sytuacji kiedy można polegać tylko na danych z jednego urządzenia. Z uwagi na jedną ze wskazanych motywacji podjęcia przez autora badań na śledzeniem ruchu było zastosowanie opracowanej metody w rehabilitacji ruchowej, ruchy były wykonywane we właściwym, dla tego obszaru zastosowań, tempie. Testy zostały wykonane na przykładzie ruchów ręki, jednakże nie ogranicza to uniwersalności zaproponowanego rozwiązania.

Wyniki uzyskane za pomocą autorskiej metody zostały porównane z metodą opisaną w literaturze o najwyższej deklarowanej dokładności na podstawie 3 parametrów: wartości błędu szacowania pozycji stawu łokciowego, wartości błędu szacowania pozycji stawu nadgarstkowego oraz wartości błędu szacowania wartości kąta zgięcia ręki w łokciu. W przypadku tego porównania, błąd został zdefiniowany jako uśredniony błąd średniokwadratowy wyliczany ze wszystkich powtórzeń danego ruchu testowego. Eksperymenty wykazały poprawę wartości badanych parametrów wyznaczonych za pomoca metody autorskiej w porównaniu do metody zaczerpniętej z literaturny maksymalnie o 18\% dla stawu łokciowego, 16\% dla nadgarstkowego i 11\% dla kąta zgięcia ręki w łokciu.

W związku z uzyskanymi wynikami, wykazane zostało, że śledzenie ruchu człowieka za pomocą systemu hybrydowego łączącego dane z kontrolera Kinect i czujników inercyjnych, może być realizowane z powodzeniem w sposób alternatywny do metod opisanych w literaturze.

Opacowana i przedstawiona metoda może być przedmiotem dalszych badań i rozwoju. Autor zidentyfikował kierunki, w jakich taki rozwój wydaje się być zasadny:

\begin{enumerate}
	\item uwzględnienie charakterystyki błedów działania i ich kompensacji dla kontrolera Kinect 2.0 działającego w oparciu o inną technologię niż wersja wykorzystana w niniejszej pracy;
	\item poszerzenie spektrum ruchów o te charakterystyczne dla dyscyplin sportowych odznaczających się dużą dynamiką zmian. To z kolei może mieć wpływ na konieczność przedefiniowania kryteriów wykorzystywnych przy łączeniu danych ze względu na większą fragmentację otrzymanych danych;
	\item odporność proponowanego hybrydowego systemu śledzenia ruchu na warunki panujące w pomieszczeniu gdzie odbywa się śledzenie. W trakcie prowadzonych badań, niejednokrotnie zdarzyło się że musiały być one przerwane ze względu na zakłócenia jakim poddane były urządzenia pomiarowe, a przez to uszkodzenia danych pomiarowych. Wobec tego zasadnym jest opracowanie metody zabezpieczania i naprawienia, uszkodzonych w trakcie transmisji, danych;
	\item zmiana architektury autorskiego urządzenia agregującego sygnały z czujników inercyjnych z obecnej opartej na jednym module centralnym i kilku czujników inercyjnych podłączonych do niego (rys. 3.3, str. 84), na rzecz zestawu autonomicznych modułów inercyjnych wysyłających samodzielnie dane do komputera przetwarzającego. Taka zmiana architektury wspomnianego urządzenia mogłaby jednak spowodować problemy synchronizacji sygnałów pomiedzy nawet kilkunastoma urzadzeniami pomiarowymi, co mogłoby wprowadzic istotne błedy w sledzeniu ruchu postaci. Z tego zaś powodu wymagałaby opracowania osobnej metody niwelowania tych problemów.
\end{enumerate}



	\end{FlushLeft}
\end{document}